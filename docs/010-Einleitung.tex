\chapter{Einleitung}
Die fortschreitende Digitalisierung sowie die zunehmende Verdichtung von Arbeitsprozessen 
stellen Unternehmen vor neue Herausforderungen im Bereich des betrieblichen 
Gesundheitsmanagements. Insbesondere in technologieintensiven Berufsfeldern, die durch hohe 
kognitive Anforderungen und komplexe Problemlösungsaufgaben geprägt sind, 
gewinnt die psychische Gesundheit der Beschäftigten zunehmend an Bedeutung. Vor 
diesem Hintergrund implementieren Organisationen verstärkt interne Präventionsprogramme. Deren 
Erfolg hängt jedoch entscheidend davon ab, wie die erhobenen Daten interpretiert und in 
konkrete, zielgerichtete Maßnahmen überführt werden können. Die vorliegende Arbeit beschäftigt 
sich mit der analytischen Aufbereitung und Auswertung solcher Daten unter Anwendung 
fortgeschrittener Methoden des Data Science.

\section{Problemstellung}
Der untersuchte Datensatz basiert auf einer umfangreichen Befragung mit über 60 Fragen. Die 
praktische Nutzung dieser Daten ist jedoch mit mehreren Herausforderungen verbunden. 
Zum einen führt die Vielzahl erhobener Variablen zu hoher Dimensionalität, wodurch die 
Identifikation relevanter Einflussfaktoren erschwert wird. Zum anderen sind im Datensatz 
fehlende Werte enthalten, die nicht ohne Weiteres ignoriert werden können. Zusätzlich liegen 
qualitative Rückmeldungen der Mitarbeiter in Form unstrukturierter Textdaten vor, die sich mit 
klassischen statistischen Verfahren nicht direkt skalieren und enkodieren lassen.

\section{Zielsetzung}
Ziel dieser Arbeit ist es, den Datensatz durch geeignete Preprocessing- und Analyseverfahren so 
aufzubereiten, dass er als valide Entscheidungsgrundlage für die HR-Abteilung
dient. Im Mittelpunkt steht dabei die Transformation komplexer Rohdaten in interpretierbare und 
relevante Informationen.

\section{Vorgehensweise}
Die methodische Struktur der Arbeit folgt einem klassischen Data-Science-Workflow, der sich in 
sechs aufeinanderfolgende Phasen gliedert. Den Ausgangspunkt bildet die Explorative Datenanalyse, 
um ein Verständnis für Verteilungen und Korrelationen zu gewinnen. Darauf folgt die 
Datenbereinigung, insbesondere die Imputation fehlender Werte. Im Schritt des Feature Engineering 
werden relevante Merkmale ausgewählt und gegebenenfalls neue Features generiert. Der 
Kern der Analyse besteht aus der Dimensionsreduktion, um den Datenraum zu komprimieren, gefolgt 
vom Clustering, um Muster zu segmentieren. Abschließend erfolgt die Interpretation der Cluster 
und die Visualisierung der Ergebnisse, um konkrete Handlungsempfehlungen für das 
Gesundheitsmanagement zu formulieren.
