\chapter{Datenbeschreibung und Explorative Datenanalyse}
Im Rahmen der Exploratory Data Analysis wurde der Datensatz auf Struktur, 
Verteilungen, fehlende Werte und potenzielle Inkonsistenzen untersucht. 
Dabei wurden zentrale Merkmale analysiert und erste Muster identifiziert, 
die Hinweise auf relevante Einflussfaktoren psychischer Belastung liefern. 
Die Ergebnisse der EDA bilden die Grundlage für die anschließende Vorverarbeitung 
und das Feature Engineering.

\section{Herkunft und Struktur des Datensatzes}
- Quelle (z. B. Kaggle OSMI Mental Health in Tech 2016)\\
- Stichprobe beschreiben, Anzahl der Merkmale, Datentypen\\
- Besonderheiten: Freitextfelder, kategoriale Felder, sensible Daten
\begin{itemize}
    \item Also Quelle ist https://www.kaggle.com/datasets/osmi/mental-health-in-tech-2016?resource=download
    \item Die OSMI Mental Health in Tech Survey 2016 ist eine internationale Umfrage mit 
über 1400 Teilnehmern aus dem IT- und Tech-Bereich. Ziel der Umfrage ist es, 
Einstellungen gegenüber psychischer Gesundheit am Arbeitsplatz zu erfassen und 
die Häufigkeit psychischer Erkrankungen unter Beschäftigten in der Tech-Branche 
zu untersuchen. Die gesammelten Daten werden vom Open Sourcing Mental Illness (OSMI) 
Team genutzt, um das Bewusstsein für psychische Gesundheit zu stärken und die 
Arbeitsbedingungen für Betroffene in der IT zu verbessern.
    \item 1433 Zeilen also Teilnehmer und 63 Spalten also Fragen. [9]
    \item Datentypen sind int64, float64 und hauptsächlich object also eine Texteingabe. [7]
    \item Fehlende Werte gibts i.d.R. viele (siehe Heatmap). Zwischen Fragen 0 bis 36 gibts
immer wieder Leute die nichts eingetragen haben. Die meisten fehlende Werte liegen [43]
zwischen Frage 16 und 24. Das sind Fragen wie:\\
- "Do you have medical coverage 
(private insurance or state-provided) which includes treatment of  
mental health issues?\\
- If you have been diagnosed or treated for a mental health disorder, do you 
ever reveal this to coworkers or employees?\\
- Do you believe your productivity is ever affected by a mental health issue?\\
Also konkrete und stark private Fragen, die jedoch am meisten zum Thema beitragen
\end{itemize}

\section{Erste deskriptive Analysen???}
- Verteilungen wichtiger Merkmale (Gender, Age, Wohnland/Arbeitsland)\\
- Identifikation möglicher Probleme: Outlier, Inkonistenzen\\
\begin{itemize}
    \item es wurden basisdaten analysiert wie GENDER [37], AGE[38], WOHNLAND[40]
     und Arbeitsland[41]
    \item Da GENDER keine vordefinierte Antworten hatte, hat jeder Befragte seine eigene 
Antwort geschrieben, was dazu führt dass danach eingie Gruppen zusammengeführt werden
müssen (wie Female, female, f) und andere komplett entfernt (z.B. Dude, mail)
    \item beim AGE gibts Ausreißer [39]über 100 und 300 Jahre analysiert
    \item Arbeitsland und Wohnland sehen gut aus, hier zeigt sich dass der Großteil
aus USA und UK kommt
    \item Die Befragten arbeiten hauptsächlich in dem Land in dem sie wohnen (1407:26)
\end{itemize}



\chapter{Datenvorverarbeitung}

\section{Umgang mit fehlenden Werten}
- Identifikation der fehlenden Werte (siehe Heatmap)\\
- Strategien (z. B. Dropping, Imputation, Domain-Knowledge)\\
- Begründung der gewählten Methode\\
\begin{itemize}
    \item zuerst werden alle offenen Fragen gelöscht die schwer von KI zu
interpretieren sind für eine Clustering Aufgabe (z.B. Why or why not?) und Fragen
die auf vorherige Antwort bezogen sind also (What US state do you work in/live in?) [114]
    \item bezüglich fehlenden Werten werden zuerst Befragte gesucht die von allen
Fragen mehr als 40\% unbeantwortet haben. Diese entfernen! [115]
    \item jetzt die Fragen die mehr als 40\% missing ratio haben werden gelöscht [117]
    \item Befragte die weniger als 25\%fehlende Werte haben, werden durch Imputation
ergänzt. Hier werden mehrere Antwortarten und Imputationen unterschieden: [119, 122]
\\ - kategorial (ja/nein/idk) -- idk selbst wählen
\\ - ordinal (gut/mittel/schlecht) -- median(), zuerst in zahlen kodieren, dann
Median berechnen und dann entkodieren
\\ - numerisch (z.B. Alter) -- median()
\\ - multilabel (z.B. Job-Rollen) -- "UNKNOWN"
\\ - binary (0/1. ja/nein) -- mode()
\\ Es wurden insgesamt 1014 und davon wurden nur die berücksichtigt die tatsächlich
fehlende werte haben. Von den Befragten (missing ratio < 0.25) wurden dann 3 Fragen
gefunden die fehlende Werte hatten
    \item Frage 4: kategorial: aus no/notsure/yes, wird automatisch der wert notsure
zugewiesen
    \item Frage 32: auch kategorial: aus no/maybenotsure/yesiobserved/yesiexperienced
wird automatisch maybe/notsure zugewiesen 
    \item Frage 41: sollten eigentlich drei vordefinierte Werte sein - male, female, 
others. Dann fehlen nur 3 Werte und es werden einfach zu Others hinzugewiesen.
Der Umgang mit verschiedenen Kategorien (also Vereinheitlichen) wird im nächste
Kapitel beschrieben.
\end{itemize}

\section{Ausreißer und Werte vereinheitlichen}
- Vereinheitlichung von Kategorien\\
- Lowercasing, Mapping, Domain-basierte Zusammenführung\\
- Umgang mit Freitext-Antworten\\
\begin{itemize}
    \item jeder hat freien text geschrieben und es sind antworten gekommen wie "f", "cis man"
"none of your business", deswegen bevor die fehlenden werte imputiert werden, müssen
die antworten vereinheitlicht werden (durch mapping). Es wurden keywords vordefiniert
z.b. bei man (male, m, man, ...). Falls solche in der Antwort vorkommen, wird es zu Man 
gemappt, genauso mit Female. Alles andere wird zu Others zugewiesen [119]
    \item beim age werden ausreißer und unseriöse werte. Allgemein werden also Befragte
mit Alter <17 und >67 gelöscht [124]
    \item Einzelwerte von AGE wurden in Gruppen zusammengefasst (17-25, 26-35,
    36-45, 46-55, 56-67) [124]
    \item länder (WOHNLAND und ARBEITSLAND) werden zu 10 am häufigsten vorkommenden
Ländern zusammengefasst, der Rest zu Others [124]
    \item bei Jobrollen gibts sehr viele Angaben (auch mehrere Rollen pro Befragten)
Diese werden zu übergeordneten Gruppen zugewiesen (z.B. Backend/Frontend Developer 
einfach zu Developer). es werden 8 Hauptgruppen unterschieden; Management/Lead, Developer,
DevOps, Product Design, Data \& Analytics, HR/Admin, Community, Other. Da viele Personen
mehrere "Rollen" haben wird noch eine Priorität bei der Auswahl gewählt: Lead=1, Developer=2,
usw. bis Community=7 und Other=99
\end{itemize}

\section{Kodierung und Transformation der Merkmale???}
- One-Hot-Encoding, Ordinal Encoding, ggf. Target-Encoding\\
- Skalierung (Transformation)\\
- Herausforderungen bei hochkardinalen Features\\
\begin{itemize}
    \item binäre Daten - OneHot, kategorial nominale - OneHot, kategoriale ordinale - 
OrdinalEncoder
    \item für binäre/nominale daten wurden zuerst manuell alle antworten analysiert 
und nur die spalten gewählt die diese art von antworten haben, danach transformiert
fitten.
    \item ordinale sind alle restlichen (werte wie "I dont know werden immer ganz hinten
liegen, modelle können es dann als 'separate kategorie berachten'. es wird nicht gelöscht,
da viele personen solche antwort gewählt haben und diese kann sehr wertvoll sein)
    \item für onehot wurde neues dataframe erstellt, für die ordinale wurde data\_clean 
überschrieben. 
    \item Mit pd.concat() wurden die beiden kombiniert zu einem neuen dataframe
data\_encoded, welches nur aus numerischen werten besteht
    \item ordinalencodierte und onehotencodierte skalieren wegen Einheitlichkeit
für abstandbasierte ML-Modelle. Dies erfolgt mit StandardScaler aus 
sklearn.preprocessing. Es wird nicht mit MinMax gemacht, da ...
    \item der vorverarbeitete datensatz wurde als neue datei gespeichert
data\_preprocessed.csv
\end{itemize}


\chapter{Feature Engineering}

\section{Feature Selection}
- Variance Threshold\\
- Korrelationen / Redundanz\\
- Relevanzbasierte Auswahl (Mutual Information)\\
\begin{itemize}
    \item für unsupervised learning eignen sich am besten der Variance Threshold und
die Korrelationsmatrix
    \item zuerst variance threshol mit einem schwellenwert von 0.01. Es wurden zwei 
Fragen gefunden die binär waren und nur eine Antwort haben\\
- Are you self-employed?\_0 mit allen 0 bzw False Werten(\_1 gibts nicht, da alle 
NEIN beantwortet haben)\\
- Do you have previous employers?\_1  mit allen 0 bzw False (\_0 gibts nicht, da alle
auch NEIN beantwortet haben)\\
also kann man beide löschen, da sie immer die gleichen werte haben
    \item es wurden bei korrelationsmatrix mehrere fragen automatisch gewählt.
    \item - Is your employer primarily a tech company/organization?\_0.0\\
- Is your employer primarily a tech company/organization?\_1.0: 1.00\\
Es ist eine binäre Frage die one-hot encodiert ist, und somit wenn ich eine davon lösche,
verliere ich keine Informationen von den anderen\\
das gleiche gilt für\\
- Have you heard of or observed negative consequences for co-workers who have been open about mental health issues in your workplace?\_No\\
- Have you heard of or observed negative consequences for co-workers who have been open about mental health issues in your workplace?\_Yes: 1.00\\
und noch diese\\
- Have you been diagnosed with a mental health condition by a medical professional?\_No\\
- Have you been diagnosed with a mental health condition by a medical professional?\_Yes: 1.00\\
und diese
- Have you ever sought treatment for a mental health issue from a mental health professional?\_0\\
- Have you ever sought treatment for a mental health issue from a mental health professional?\_1: 1.00\\
    \item bezüglich Wohnland und Arbeitsland wird die jede Spalte vom Wohnland entfernt,
da die kleinste Korrelation einen wert von 0.96 beträgt, was sehr hoch Inkonistenzen
    \item AUF FEATURE GENERATION WURDE VERZICHTET, WEIL...
\end{itemize}

\section{Methoden der Dimensionsreduktion}
- PCA (linear)\\
- MDS, LLE (nichtlinear)\\
- Vergleich und Begründung der Auswahl\\
\subsection{ANSATZ A: PCA}
\begin{itemize}
    \item WARUM DIMENSIONSREDUKTION?
    \item PCA wird verwendet, weil sie auf varianz basiert und alle werte sind bereits
encodiert und standardisiert auf mittelwert 0 und varianz 1
    \item die Features wurden linear skaliert mit StandardScaler, daher sind die für
lineare Analyse gut vorbereitet (PCA ist linear)
    \item MDS nicht, weil distanzbasiert und ich muss zuerst ein Distanzmaß definieren
    \item LLE nicht, weil setzt Mannigfaltigkeiten voraus, das trifft bei Fragebogendaten
fast nie zu, benötigt viele Datenpunkte
    item zuerst wird ein Explained Variance Plot dargestellt um die beste Anazahl von
Hauptkomponenten zu finden
    \item der Ellenbogen Punkt liegt bei 10, jedoch es ist unter 85\% Varianz, diese
befindet sich erst bei 64 Hauptkomponenten. 10 enthält nur 32\% der Varianz. Es wird
also K=64 gewählt.
    \item DIAGRAMME ZEIGEN
    \item PROBLEM: sehr viele Hauptkomponenten, schwierig interpretierbar, UND wenn
Anzahl Cluster Berechnung mit BIC ist ein sehr hoher Wert (kleinster bei k=2 220000)
und SilhouettenScore sehr niedrig (größter bei k=2 0.044)
    \item aus diesen Gründen wurde Ansatz B gewählt
\end{itemize}
\subsection{ANSATZ B: manuelle Feature Transformation}
    \begin{itemize}
    \item hierfür wurden Fragen die zu gleichen Themen gehören zusammengefasst, 
Antworten gewichtet, in neuer Feature gespeichert und danach gelöscht
    \item NEUE Features:
    \item employer\_support\_score - höher je mehr Unterstützung der Arbeitgeber bietet
    \item prev\_employer\_support\_score - höher je mehr Unterstützung der ehemalige 
Arbeitgeber geboten hat
    \item openness\_score - wie offen der Befragte bei einem neuen Arbeitgeber wäre
    \item perceived\_stigma\_score - höher, wenn der Befragte denkt, dass MH schadet
(Karriere/Team)
    \item mh\_status\_score - misst wie stark der Befragte mental gesund ist
    \item jede Antwort wird entsprechend bewertet, werte die gegen den Score sind MINUS
und Werte für den Score sind PLUS. Danach wird die Summe aller Fragen berechnet, die
zu dem Score gehören und der Durschnitt berechnet (danach mit StandardScaler skaliert)
    \item insgesamt 106 Fragen zu 5 Features zusammengefasst und danach Fragen gelöscht
    \item VORTEILE: jetzt sind es insgesamt 35 Features, also weniger als 64 bei PCA,
die Features sind gut interpretierbar und außerdem haben die besseren BIC (k=2 35000)
und silhouetten (k=2 0.29)
    \item bic sinkt mit anzahl cluster (also laut bic k=12 besser als k=2), beim
silhouetten ist umgekehrt. Es werden k=3 gewählt (also eher silhouetten betrachten), 
da es angibt wie gut getrennt und ziel der aufgabe ist gute interpretierbarkeit.
Es wird k=3 gewählt, da laut bic je mehr desto besser und bei silhouetten ist 
k=2 (0.29) und k=3 (0.28) sehr ähnlich, k=4 (0.13) sinkt sehr stark, deswegen ist 
k=3 perfekt
    \end{itemize}



\chapter{Clustering}

\section{Auswahl geeigneter Methoden}
- K-Means\\
- Agglomeratives Clustering\\
- DBSCAN/HDBSCAN für komplexe Strukturen\\
- Begründung der Auswahl\\
\begin{itemize}
    \item berücksichtigt wurden folgende Algorithmen: k-Means, GMM, DBSCAN und
hierarchisch Agglomerativ
    \item VERGLEICH QUELLEN: 
    \item https://learninglabb.com/clustering-algorithms-in-machine-learning/
    \item https://krishnapullak.medium.com/guide-to-clustering-algorithms-strengths-weaknesses-and-evaluation-5285a75ea902
    \item https://www.geeksforgeeks.org/data-science/choosing-the-right-clustering-algorithm-for-your-dataset/
    \item es wurde GMM gewählt, da KMEANS dafür zu "fest" ist, andere waren für die
Datensätze nicht geeignet.
    \item 
\end{itemize}

\section{Bestimmung der Clusteranzahl}
- Elbow-Methode\\
- Silhouette Score\\
- Weitere Metriken\\
\begin{itemize}
    \item wie vorher erwähnt durch vergleich zweier ansätze - pca und manuell generiert,
wurden bereits die k berechnet, es wird k=3 gewählt
\end{itemize}

\section{Ergebnisse}
- Visualisierungen der Cluster (PCA/UMAP Scatterplots)\\
- Profiling: Beschreibung der typischen Merkmale jedes Clusters\\
- Identifikation gefährdeter Gruppen und Muster\\
\begin{itemize}
    \item es entstehen 3 cluster (0: 102 Personen, 1: 898 Personen, 2: 9 Personen)
    \item diese wurden standardisiert:
    \item employer\_support\_score
\\Cluster 0:  0.62  (überdurchschnittlich)
\\Cluster 1:  0.54  (überdurchschnittlich)
\\Cluster 2: -1.2  (stark unterdurchschnittlich)
    \item 2 fühlt sich von Arbeitgeber sehr wenig unterstützt, 0 besser als 1 unterstützt
    \item prev\_employer\_support\_score
\\Cluster 0:  0.19   (durchschnittlich)
\\Cluster 1:  0.89   (viel support)
\\Cluster 2: -1.1   (deutlich weniger support)
    \item 2 hatte früher schlechte Arbeitgeber, 0 durchschnittliche Erfahrung, 
    1 gute erfahrung
    \item 2 zeigt klares Muster von schlechter Umgebung (früher + aktuell)
    \item openness\_score
\\Cluster 0: -0.69  (leicht weniger offen)
\\Cluster 1: -0.46  (wenig offen)
\\Cluster 2:  1.1  (deutlich offener)
    \item 2 viel offener, 0 eher verschlossen, 1 wenig offen
    \item perceived\_stigma\_score
\\Cluster 0: -0.83 (viel weniger)
\\Cluster 1: -0.29 (wenig)
\\Cluster 2:  1.1 (mehr Stigma)
    \item 2 viel stigma, 0 und 1 wenig
    \item mh\_status\_score
\\Cluster 0: -1.1  (schlechtester mental state)
\\Cluster 1:  0.36  (leicht überdurchschnittlich)
\\Cluster 2:  0.77  (bester mental state)
    \item 0 schlecht mental gesund, 2 guter zustand, 1 durchschnittlich
    \item GESAMT: (aus diesen 5 features erkennen wir) 
    \item CLUSTER 2: offene, mental stabile Personen trotz geringer Arbeitgeberunterstützung,
aber höhere stigma, also denken dass sie benachteilgt wären wenn sie mental krank wären
    \item CLUSTER 0: verschlossen, schlechter mentaler Zustand, trotz besseren support
als Cluster 2
    \item CLUSTER 1: dient eher als Vergleichsgruppe, alles durchschnittlich
    \item BILD!
    \item JOB ROLES:
    \item CLUSTER 0: viele aus community (writer, advocate), hr/admin und design,
und paar devops
    \item CLUSTER 1: eher devops paar developer und Lead
    \item CLUSTER 2: Developer und others
    \item ARBEITSLAND:
    \item CLUSTER 0: hauptsächlich Australia, Ireland aber auch Kanada und bisschen
UK, USA
    \item CLUSTER 1: Hauptsächlich Frankreich, Deutschland, Netherlands, Others, Schweden
UK, USA, bisschen auch Kanada
    \item CLUSTER 2: hauptsächlich Brazil
    \item TECH COMPANY?:
    \item CLUSTER 0: ja
    \item CLUSTER 1: eher ja aber nicht immer
    \item CLUSTER 2: nein
    \item GENDER:
    \item CLUSTER 0: hauptächlich other aber auch frauen, männer nicht
    \item CLUSTER 1: männer und frauen (frauen bissl mehr)
    \item CLUSTER 2: hauptsächlich männer
\end{itemize}

\section{Übertragung auf HR-Kontext}
- Welcher Cluster ist besonders belastet?\\
- Welche Kombinationen von Faktoren treten gehäuft auf?\\
- Welche Gruppen könnten gezielte Unterstützung benötigen?\\













