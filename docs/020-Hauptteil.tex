\chapter{Datenbeschreibung und Explorative Datenanalyse}
Im Rahmen der explorativen Datenanalyse (EDA) wurde der Datensatz hinsichtlich seiner Struktur, 
Verteilungen, fehlenden Werte sowie möglicher Inkonsistenzen untersucht. Dabei wurden zentrale 
Merkmale betrachtet und erste Muster identifiziert, die auf relevante Einflussfaktoren 
psychischer Belastung hinweisen. Die Ergebnisse der EDA bilden die Grundlage für die 
anschließenden Schritte der Datenvorverarbeitung und des Feature Engineerings. Zur besseren 
Nachvollziehbarkeit wird der zugehörige Code in diesem Dokument über den 
folgenden Verweismechanismus referenziert: \texttt{[FILEINDEX\_CODELINE]}. Ein Eintrag 
wie \texttt{[0\_31]} verweist beispielsweise auf die Datei \texttt{0\_explorative\_analysis.ipynb} 
und die dortige Codezeile 31. Der Quellcode ist auf
\href{https://github.com/thedawid999/Health-Analysis/tree/main}{GitHub} zu finden.

\section{Herkunft und Struktur des Datensatzes}
Der verwendete Datensatz stammt aus der folgenden Quelle:
\href{https://www.kaggle.com/datasets/osmi/mental-health-in-tech-2016?resource=download}{Kaggle}.
Er basiert auf der OSMI Mental Health in Tech Survey 2016, einer internationalen Befragung mit 
über 1.400 Teilnehmenden aus der IT- und Tech-Branche. Ziel der Umfrage ist es, Einstellungen 
gegenüber psychischer Gesundheit am Arbeitsplatz zu erfassen sowie die Prävalenz psychischer 
Erkrankungen unter Beschäftigten in technischen Berufen zu untersuchen.

Der Datensatz umfasst insgesamt 1433 Zeilen, die jeweils einen Teilnehmenden repräsentieren, 
sowie 63 Spalten, die unterschiedlichen Fragen entsprechen \texttt{[0\_11]}. Die Datentypen setzen 
sich aus \texttt{int64}, \texttt{float64} und überwiegend \texttt{object}-Typen zusammen, wobei 
Letztere hauptsächlich Freitexteingaben enthalten \texttt{[0\_9]}. Insgesamt weist der Datensatz 
einen vergleichsweise hohen Anteil an fehlenden Werten auf. 
\begin{figure}[H]
    \centering
    \includegraphics[width=0.9\textwidth]{pics/missing_values.png}
    \caption{Fehlende Werte des Datensatzes}
    \label{fig:missing_values}
\end{figure}
\noindent Fehlende Einträge treten bereits in den Fragen 0 bis 36 regelmäßig auf, 
konzentrieren sich jedoch besonders stark im Bereich der Fragen 16 bis 24. 
Dabei handelt es sich überwiegend um sensible und persönliche Themen wie:
“Do you have medical coverage (private insurance or state-provided) which includes treatment 
of mental health issues?” oder “If you have been diagnosed or treated for a mental health disorder, 
do you ever reveal this 
to coworkers or employees?”. Diese Fragen betreffen sehr private Aspekte der psychischen 
Gesundheit. Obwohl sie für die Fragestellung der Arbeit hohe Relevanz aufweisen, ist es 
nachvollziehbar, dass viele Teilnehmende sie unbeantwortet ließen.

\section{Erste deskriptive Analysen}
Im ersten Schritt wurden grundlegende demografische Variablen analysiert, darunter Geschlecht 
\texttt{[0\_14]}, Alter \texttt{[0\_15]}, Wohnland \texttt{[0\_40]} und Arbeitsland 
\texttt{[0\_41]}. Da die Variable Gender keine vordefinierten Antwortoptionen enthielt, konnten 
Teilnehmende beliebige Freitextangaben machen. Dies führte zu zahlreichen Varianten 
(z. B. Female, female, f) sowie zu unbrauchbaren Einträgen wie "Dude" oder "mail". Für eine 
konsistente Auswertung war daher eine nachträgliche Bereinigung und Zusammenführung der 
Kategorien erforderlich.
\begin{figure}[H]
    \centering
    \includegraphics[width=0.8\textwidth]{pics/gender_values_initially.png}
    \caption{Eingabe des Geschlechts}
    \label{fig:gender_values_intially}
\end{figure}

\noindent Bei der Variable Age traten deutliche Ausreißer auf, unter anderem Werte von über 100 bzw. 
sogar 300 Jahren \texttt{[0\_16]}. Diese wurden im Rahmen der Datenbereinigung gesondert 
betrachtet.
\begin{figure}[H]
    \centering
    \includegraphics[width=0.8\textwidth]{pics/age_outliers.png}
    \caption{Ausreißer beim Alter}
    \label{fig:age_outliers}
\end{figure}

\noindent Die Angaben zu Wohn- und Arbeitsland zeigen eine stabile und plausible Verteilung: Der Großteil 
der Befragten stammt aus den USA und dem Vereinigten Königreich. Zudem arbeiten die meisten 
Teilnehmenden in dem Land, in dem sie auch wohnen – 1407 Personen gaben an, im Wohnland zu 
arbeiten, während 26 Personen im Ausland tätig sind \texttt{[0\_23]}.



\chapter{Datenvorverarbeitung}

\section{Umgang mit fehlenden Werten}
Zu Beginn wurden alle offenen Freitextfragen entfernt, deren Inhalte für Clustering schwer 
interpretierbar sind (z. B. “Why or why not?”) sowie Fragen, die sich ausschließlich auf 
vorherige Antworten beziehen, etwa “What US state do you work in/live in?” 
\texttt{[1\_4]}. Im Anschluss wurden fehlende Werte systematisch behandelt. Zunächst wurden 
alle Befragten gelöscht, die mehr als 40\% der Fragen unbeantwortet ließen \texttt{[1\_5]}. 
Anschließend wurden alle Fragen (Spalten) entfernt, deren Missing Ratio ebenfalls über 40\% lag 
\texttt{[1\_7]}.

Für die verbleibenden Teilnehmenden mit weniger als 25\% fehlenden Werten erfolgte eine 
Imputation \texttt{[1\_9]}. Insgesamt enthielt der Datensatz 1014 Fragen, von denen jedoch 
nur diejenigen berücksichtigt wurden, die tatsächlich fehlende Werte aufwiesen. Für die Gruppe 
der Befragten mit einer Missing Ratio unter 0.25 wurden dabei lediglich drei Fragen 
identifiziert, die eine Imputation benötigten \texttt{[1\_12]}:
\begin{itemize}
    \item Frage 4: Antwortmöglichkeiten \textit{no / not sure / yes}. Fehlende Werte wurden 
    einheitlich mit \textit{not sure} imputiert.
    \item Frage 32: Antwortmöglichkeiten \textit{no / maybe-notsure / yes I observed / yes 
    I experienced}. Fehlende Werte wurden auf \textit{maybe-notsure} gesetzt.
    \item Frage 41: Ursprünglich drei Kategorien \textit{male, female, others}. Da lediglich 
    drei Werte fehlten, wurden diese der Kategorie \textit{others} zugewiesen.
\end{itemize}
\noindent Die Vereinheitlichung der verschiedenen kategorischen Antworten für das Geschlecht
wird im folgenden Kapitel detailliert beschrieben.

\section{Ausreißer und Werte vereinheitlichen}
Da viele Fragen als Freitexteingaben formuliert waren, traten zahlreiche uneinheitliche oder 
informelle Antworten auf, etwa “f”, “cis man” oder “none of your business”. Bevor fehlende 
Werte imputiert werden konnten, war daher eine umfassende Vereinheitlichung der Kategorien 
erforderlich. Dies erfolgte über ein regelbasiertes Mapping: Für die Geschlechtsangabe wurden 
beispielsweise Schlüsselwörter definiert (z. B. male, m, man). Enthielt eine Antwort eines 
dieser Keywords, wurde sie der Kategorie Male zugeordnet; entsprechend wurde mit Begriffen rund 
um Female verfahren. Alle übrigen Eingaben wurden der Kategorie Others zugewiesen. Die Umsetzung 
erfolgte mithilfe regulärer Ausdrücke (Regex) \texttt{[1\_10]}.

Für die Variable Age wurden offensichtliche Ausreißer ausgeschlossen. Teilnehmende unter 17 oder 
über 67 Jahren wurden als nicht plausibel eingestuft und entfernt. Anschließend wurden die 
validen Altersangaben in fünf Gruppen kategorisiert: \textit{17–25}, \textit{26–35}, 
\textit{36–45}, \textit{46–55} und \textit{56–67} \texttt{[1\_14]}.

Die Variablen Wohnland und Arbeitsland wurden zu den zehn am häufigsten vorkommenden Ländern 
zusammengefasst. Alle weiteren Länder wurden in der Kategorie Others gebündelt \texttt{[1\_15]}.

Bei den Jobrollen zeigte sich eine besonders große Heterogenität, da viele Befragte mehrere 
Rollen gleichzeitig angaben und die Bezeichnungen stark variierten. Daher wurden alle Angaben 
in übergeordnete Hauptgruppen überführt: \textit{Management/Lead}, \textit{Developer}, 
\textit{DevOps}, \textit{Product Design}, \textit{Data \& Analytics}, \textit{HR/Admin},
\textit{Community} und \textit{Other}. Da einzelne Personen mehrere Rollen nannten, 
wurde zusätzlich ein Prioritätensystem eingeführt (z. B. Lead = 1, Developer = 2, …, Community = 7, 
Other = 99), um eine eindeutige Zuordnung zu gewährleisten \texttt{[1\_16]}.

\section{Kodierung und Transformation der Merkmale}
Nachdem die fehlenden Werte bereinigt und kategoriale Angaben vereinheitlicht wurden, konnte 
die eigentliche Merkmalskodierung durchgeführt werden. Je nach Skalenniveau der Variablen wurden 
unterschiedliche Verfahren angewendet

Binäre und nominale Merkmale wurden mittels One-Hot-Encoding transformiert.
Vor der Kodierung wurden alle Fragen manuell identifiziert, deren Antwortmöglichkeiten 
ausschließlich binär oder nominal ausgeprägt sind. Nur diese Spalten wurden anschließend in den 
One-Hot-Encoder übergeben. Dies verhindert, dass ordinale oder numerische Variablen 
fälschlicherweise als nominal behandelt werden. Der Encoder wurde auf den relevanten Spalten 
gefittet und auf die Daten angewendet \texttt{[1\_18]}.

Alle verbleibenden kategorialen Merkmale wurden als ordinal betrachtet und mithilfe des 
\textit{OrdinalEncoder} kodiert. Dabei wurde die Reihenfolge der Kategorien anhand ihrer tatsächlichen 
semantischen Struktur bzw. durch eine sinnvolle numerische Reihenfolge festgelegt.
Antwortoptionen wie “I don’t know”, “Not applicable to me” oder “Unsure” wurden bewusst nicht 
entfernt, sondern stets als letzte Kategorie eingeordnet, sodass Modelle diese Werte eindeutig 
als separate Kategorie erkennen können. Diese Werte wurden nicht gelöscht, da sie von vielen 
Befragten gewählt wurden und potenziell wichtige Informationen enthalten \texttt{[1\_19]}.

Die Ergebnisse beider Kodierschritte wurden in einem neuen DataFrame data\_encoded 
zusammengeführt, der alle transformierten Merkmale (One-Hot-Variablen und ordinal kodierte 
Spalten) umfasst. Der vollständig vorverarbeitete Datensatz wurde anschließend als 
data\_preprocessed.csv gespeichert, um eine reproduzierbare Weiterverarbeitung in den nächsten 
Analyse- und Clustering-Schritten zu gewährleisten [1\_20].



\chapter{Feature Engineering}

\section{Feature Selection}
Für Unsupervised Learning eignen sich insbesondere der Variance Threshold sowie die 
Korrelationsmatrix (\cite{scikit_learn_feature_selection}), da beide Verfahren ohne 
Zielvariable funktionieren 
und helfen redundante oder nicht-informative Merkmale aus dem Datensatz zu entfernen. Der 
Variance Threshold identifiziert Merkmale ohne Varianz, während die Korrelationsmatrix hoch 
korrelierte Feature-Paare aufdeckt, deren Informationen sich stark überschneiden.

\subsection{Variance Threshold}
Zunächst wurde ein Variance Threshold von 0,01 angewendet. Dabei zeigte sich, dass bei zwei Merkmalen 
jeweils die "Nein"-Variante (\_0) bzw. "Ja"-Variante (\_1) derselben Frage, alle Beobachtungen denselben Wert 
aufwiesen. Konkret betraf dies die Fragen ''Are you self-employed?'' und ''Do you have previous employers?'', 
bei denen ausschließlich eine der beiden Kategorien vorkam. Da bei solchen Merkmalen die zweite
Kategorie redundant ist, wurden diese Merkmale entfernt \texttt{[2\_58]}.

\subsection{Korrelationsmatrix}
Im nächsten Schritt wurde eine Korrelationsmatrix berechnet, um stark korrelierte Merkmale zu 
erkennen \texttt{[2\_59]}. Merkmale mit einer Korrelation von 1.00 wurden als redundant betrachtet. Dabei zeigte 
sich, dass bestimmte One-Hot-Encoder-Paare vollständig redundant sind: ''Is your employer primarily 
a tech company/organization?\_0.0'' und ''Is your employer primarily a tech company/organization?\_1.0''
hatten eine Korrelation von 1.00. Da es sich um eine ursprünglich binäre Variable handelt, reicht 
es aus, eine der beiden Spalten beizubehalten, ohne Informationsverlust. Das gleiche gilt für 
alle andere Merkmals-Paare, deren Korrelation 1.00 beträgt.

\subsection{Wohnland vs. Arbeitsland}
Die Analyse der Korrelationen zwischen Wohnland und Arbeitsland (beide ebenfalls One-Hot-enkodiert) 
zeigte für alle Paare extrem hohe Korrelationen, mit einem Minimum von 0.96 \texttt{[2\_60]}. Da die 
überwiegende Mehrheit der Befragten in dem Land arbeitet, in dem sie auch wohnen, enthalten 
diese Merkmale praktisch die gleiche Information. Um Redundanz zu vermeiden, wurden daher alle 
Wohnland-Merkmale entfernt, während die Arbeitsland-Merkmale beibehalten wurden.

\section{Methoden der Dimensionsreduktion}
Dimensionsreduktion vereinfacht komplexe, hochdimensionale Datensätze, indem irrelevante oder 
redundante Merkmale entfernt werden. Das reduziert Rechenaufwand und erleichtert Analyse und 
Visualisierung, ohne die wichtigsten Informationen zu verlieren. Sie hilft außerdem, Overfitting 
und den „Curse of Dimensionality“ zu mindern, sodass Modelle robuster und 
generalisierungsfähiger werden (\cite{evoluce_dimensionsreduktion}). Es wurden zwei unterschiedliche
Ansätze verwendet und verglichen, um denjenigen mit der besten Balance aus Interpretierbarkeit
und Modellqualität zu identifizieren.

\subsection{ANSATZ A: PCA}
Zunächst wurde Principal Component Analysis verwendet. PCA eignet 
sich insbesondere für diesen Datensatz, da sie varianzbasierend arbeitet und alle Merkmale 
bereits enkodiert sowie standardisiert wurden (\cite{ibm_pca}). MDS (Multidimensional Scaling) 
wurde nicht verwendet, da es distanzbasiert ist und die Wahl 
eines geeigneten Distanzmaßes bei hochdimensionalen Fragebogendaten 
problematisch ist (\cite{groenen2013past}). Auch LLE (Locally Linear Embedding) wurde nicht 
eingesetzt, da dieses Verfahren annimmt, dass die Daten auf einer nichtlinearen Mannigfaltigkeit 
liegen und viele Beobachtungen pro lokaler Nachbarschaft benötigen (\cite{roweis2000nonlinear}). Diese 
Voraussetzungen sind bei typischen Umfragedaten meist nicht gegeben.
Die Bestimmung der Anzahl der Hauptkomponenten erfolgte wie folgt: zunächst wurde die 
kumulative erklärte Varianz der PCA analysiert \texttt{[2\_65]}. Der sogenannte Ellenbogenpunkt lag bei 
etwa 10 Komponenten, jedoch deckten diese lediglich 32\% der Gesamtvarianz ab. Erst bei 64 
Hauptkomponenten wurde ein akzeptabler Wert von 85\% kumulierter Varianz erreicht. Aus diesem 
Grund wurde K = 64 als Anzahl der Hauptkomponenten gewählt.
\begin{figure}[H]
    \centering
    \includegraphics[width=0.5\textwidth]{pics/cumulative_explained_variance_pca.png}
    \caption{Kumulative Erklärte Varianz für PCA}
    \label{fig:cumulative_explained_variance_pca}
\end{figure}
\noindent Die Wahl von 64 Hauptkomponenten führte allerdings zu zwei wesentlichen Problemen. Als Erstes erschwert
eine so hohe Anzahl and Komponenten die inhaltliche Interpretation erheblich. Und als Zweites
führte diese Methode zu schwachen Clustermetriken. Bei anschließenden Clustering-Versuchen zeigten sich 
ein sehr hoher BIC und extrem niedrieger Silhouettenwert 
(höchster Wert: 0.044 bei K = 2 \texttt{[3\_65]})

\begin{figure}[H]
    \centering
    \includegraphics[width=0.5\textwidth]{pics/bic_PCA.png}
    \caption{BIC und AIC für PCA}
    \label{fig:bic_pca}
\end{figure}
\noindent Aufgrund der genannten Probleme wurde entschieden, nicht mit dem PCA-basierten Ansatz 
fortzufahren. Stattdessen wurde Ansatz B gewählt.

\subsection{ANSATZ B: manuelle Feature Transformation}
Beim zweiten Ansatz wurde der ursprüngliche, hochdimensionale Datensatz gezielt inhaltlich 
reduziert, indem thematisch zusammenhängende Fragen zu aussagekräftigen 
Merkmalen zusammengeführt wurden \texttt{[2\_70]}. Dazu wurden Antworten inhaltlich gewichtet, zu neuen 
Feature-Scores aggregiert und anschließend die ursprünglichen Einzelmerkmale entfernt.
Insgesamt wurden fünf neue, gut interpretierbare Scores gebildet:

\noindent \textbf{employer\_support\_score} misst, wie stark der derzeitige Arbeitgeber mentale 
Gesundheit unterstützt (höherer Wert = stärkerer Support),
\textbf{prev\_employer\_support\_score} ist analog zum obigen Score, jedoch für den 
vorherigen Arbeitgeber,
\textbf{openness\_score} erfasst, wie offen der Befragte gegenüber einem neuen
Arbeitgeber in Bezug auf mentale Gesundheit wäre,
\textbf{perceived\_stigma\_score} misst, wie stark der Befragte die Meinung vertritt, 
dass Offenheit über mentale Gesundheit der Karriere oder dem Team schaden könnte
(höherer Wert = stärker wahrgenommenes Stigma) und
\textbf{mh\_status\_score} bewertet den subjektiven mentalen Gesundheitsstatus der Person.
Für jeden Score wurde eine Bewertungslogik verwendet. Antworten, die gegen den Score 
sprechen bekamen negative Werte, Antworten, die für den Score sprechen bekamen positive Werte.
Pro Score wurden alle zugehörigen Fragen aufsummiert und anschließend der Durchschnitt 
berechnet \texttt{[2\_71]}. Die fünf neuen Features wurden anschließend zum Datensatz 
hinzugefügt und mit \texttt{StandardScaler} standardisiert \texttt{[2\_73]}.
Durch diese manuelle Merkmalskonstruktion ergaben sich mehrere Vorteile: Reduktion der 
Dimensionalität auf 35 Merkmale, deutlich bessere Interpretierbarkeit und verbesserte
Clustering-Metriken (BIC 35000 für K = 2 und Silhouettenwert 0.29 für K = 2)
\begin{figure}[H]
    \centering
    \includegraphics[width=0.6\textwidth]{pics/bic_man_gen.png}
    \caption{BIC und AIC für manuell generierte Features}
    \label{fig:bic_man_gen}
\end{figure}
\noindent Der BIC sinkt mit steigender Clusteranzahl, also bevorzugt es höhere k-Werte. Der Silhouetten-Score
zeigt dagegen eine klare Verschlechterung ab K $>$ 4.
\begin{figure}[H]
    \centering
    \includegraphics[width=0.2\textwidth]{pics/silhouette_scores_man_gen.png}
    \caption{Silhouetten-Score für manuell generierte Features}
    \label{fig:silhoutte_man_gen}
\end{figure}
\noindent Der BIC misst, wie gut ein Modell zu den Daten passt und bestraft unnötig viele 
Cluster. Niedrigere Werte sind besser, aber BIC neigt dazu, höhere Clusterzahlen zu bevorzugen.
Der Silhouetten-Score misst, wie gut die Cluster voneinander getrennt sind. 
Für diese Aufgabe ist er wichtiger, da das Ziel klar interpretierbare Cluster 
sind. K = 2 hätte den besten Silhouetten-Score, aber der Unterschied zu K = 3 ist minimal. Da der 
BIC auch berücksichtigt wurde und K = 3 eine etwas sinnvollere Struktur bietet, wurde 
K = 3 gewählt. 



\chapter{Clustering}

\section{Auswahl geeigneter Methoden}
Im Kontext der hochdimensionalen und heterogenen Umfragedaten erweist sich das Gaussian Mixture Model 
als am besten geeignetes Clustering-Verfahren. GMM erlaubt die Modellierung elliptischer und 
unterschiedlich großer Cluster und überwindet damit die Einschränkung von k-Means, das nur sphärische 
und gleich große Cluster zuverlässig erkennen kann. Gleichzeitig arbeitet GMM 
probabilistisch, was bei psychologischen Daten mit fließenden Übergängen zwischen Gruppen eine 
angemessene weiche Zuordnung ermöglicht (\cite{learninglabb_clustering}).
DBSCAN hingegen ist für hochdimensionale Daten problematisch, da seine Dichteparameter in solchen 
Räumen nicht mehr zuverlässig trennscharf wirken (\cite{geeksforgeeks_clustering}). Auch 
hierarchisch-agglomeratives Clustering ist aufgrund seiner hohen Rechenkomplexität und geringen 
Flexibilität bei komplexen Datensätzen weniger geeignet (\cite{pullak_clustering}).
Im Vergleich dazu bietet GMM eine höhere Modellflexibilität, bessere Anpassungsfähigkeit an die 
tatsächliche Datenstruktur und eine verständliche Clusterzuordnung. 

\section{Bestimmung der Clusteranzahl}
Wie zuvor beschrieben, wurden die geeigneten Clusterzahlen sowohl mittels PCA-basierter Analyse als 
auch durch manuell generierte Merkmalsselektionen überprüft. Beide Ansätze führten konsistent zu 
derselben Empfehlung. Auf dieser Grundlage wird die Clusteranzahl K = 3 gewählt.


\section{Ergebnisse}
Die Analyse identifiziert insgesamt drei Cluster, bestehend aus Cluster 0 (n = 102), Cluster 1 
(n = 898) und Cluster 2 (n = 9). Im Folgenden werden die zentralen Merkmale der Cluster basierend 
auf den fünf generierten Kernvariablen dargestellt.
\begin{table}[H]
    \centering
    \caption{Clustering von employer\_support\_score}
    \label{tab:clustering_ess}
    \begin{tabular}{|l|l|l|}
        \hline
        \textbf{Cluster 0} & 0.62 & überdurchschnittlich \\ \hline
        \textbf{Cluster 1} & 0.54 & überdurchschnittlich \\ \hline
        \textbf{Cluster 2} & -1.2 & stark unterdurchschnittlich \\ \hline
    \end{tabular}
\end{table}
\noindent Cluster 2 weist eine deutlich unterdurchschnittliche wahrgenommene Arbeitgeberunterstützung auf, 
während Cluster 0 und 1 leicht überdurchschnittliche Werte zeigen.

\begin{table}[H]
    \centering
    \caption{Clustering von prev\_employer\_support\_score}
    \label{tab:clustering_pess}
    \begin{tabular}{|l|l|l|}
        \hline
        \textbf{Cluster 0} & 0.19 & durchschnittlich \\ \hline
        \textbf{Cluster 1} & 0.89 & überdurchschnittlich \\ \hline
        \textbf{Cluster 2} & -1.1 & stark unterdurchschnittlich \\ \hline
    \end{tabular}
\end{table}
\noindent Cluster 2 berichtet sowohl früher als auch heute von klar negativen Erfahrungen, während Cluster 0 
durchschnittliche und Cluster 1 überdurchschnittlich positive Erfahrungen aufweist. 

\begin{table}[H]
    \centering
    \caption{Clustering von openness\_score}
    \label{tab:clustering_os}
    \begin{tabular}{|l|l|l|}
        \hline
        \textbf{Cluster 0} & -0.69 & unterdurchschnittlich \\ \hline
        \textbf{Cluster 1} & -0.46 & unterdurchschnittlich \\ \hline
        \textbf{Cluster 2} & 1.1 & stark überdurchschnittlich \\ \hline
    \end{tabular}
\end{table}
\noindent Cluster 2 zeichnet sich durch eine stark überdurchschnittliche Offenheit aus, während die Cluster 0 
und 1 jeweils unterdurchschnittliche Werte zeigen.

\begin{table}[H]
    \centering
    \caption{Clustering von perceived\_stigma\_score}
    \label{tab:clustering_pss}
    \begin{tabular}{|l|l|l|}
        \hline
        \textbf{Cluster 0} & -0.83 & stark unterdurchschnittlich \\ \hline
        \textbf{Cluster 1} & -0.29 & unterdurchschnittlich \\ \hline
        \textbf{Cluster 2} & 1.1 & stark überdurchschnittlich \\ \hline
    \end{tabular}
\end{table}
\noindent Cluster 2 weist eine deutlich überdurchschnittliche Stigmatisierung auf. Cluster 0 und 1 hingegen 
empfinden ein geringes Stigma im beruflichen Kontext.

\begin{table}[H]
    \centering
    \caption{Clustering von mh\_status\_score}
    \label{tab:clustering_mhss}
    \begin{tabular}{|l|l|l|}
        \hline
        \textbf{Cluster 0} & -1.1 & stark unterdurchschnittlich \\ \hline
        \textbf{Cluster 1} & 0.36 & leicht überdurchschnittlich \\ \hline
        \textbf{Cluster 2} & 0.77 & überdurchschnittlich \\ \hline
    \end{tabular}
\end{table}
\noindent Cluster 0 zeigt einen stark eingeschränkten mentalen Gesundheitszustand, während Cluster 2 
überdurchschnittlich und Cluster 1 leicht überdurchschnittlich abschneidet.
Es ergibt sich folgendes Clusterprofil:\\
\textbf{Cluster 2}: Offene, mental stabile Personen mit gleichzeitig geringer wahrgenommener 
Arbeitgeberunterstützung und hohem Stigmaerleben. \\
\textbf{Cluster 0}: Weniger offene Personen mit schwachem mentalem Gesundheitszustand, trotz moderater 
Arbeitgeberunterstützung. \\
\textbf{Cluster 1}: Vergleichsgruppe mit überwiegend durchschnittlichen bis leicht positiven Ausprägungen.

\begin{figure}[H]
    \centering
    \includegraphics[width=0.6\textwidth]{pics/selected_features_radar_chart.png}
    \caption{Fünf selbst-generierte Features pro Cluster}
    \label{fig:radar_chart_features_per_cluster}
\end{figure}
\noindent Das Radar-Diagramm zeigt die drei Cluster im direkten Vergleich und macht deren Profilunterschiede 
über die fünf ausgewählten Features gut sichtbar. Es verdeutlicht damit die strukturellen 
Unterschiede zwischen den Clustern auf einen Blick.

\begin{figure}[H]
    \centering
    \includegraphics[width=0.5\textwidth]{pics/job_roles_per_cluster.png}
    \caption{Arbeitsstellen pro Cluster}
    \label{fig:job_roles_per_cluster}
\end{figure}
\noindent Cluster 0 enthält vorwiegend Rollen aus Community/Advocacy, HR/Administration und Design, ergänzt 
durch einzelne DevOps-Positionen.
Cluster 1 besteht hauptsächlich aus DevOps, einigen Entwicklern und Führungskräften.
Cluster 2 umfasst überwiegend Developer sowie nicht näher spezifizierte Rollen.

\begin{figure}[H]
    \centering
    \includegraphics[width=0.5\textwidth]{pics/working_country_per_cluster.png}
    \caption{Arbeitsländer pro Cluster}
    \label{fig:working_country_per_cluster}
\end{figure}
\noindent Cluster 0 ist überwiegend in Australien und Irland vertreten, gefolgt von Kanada, dem Vereinigten 
Königreich und den USA.
Cluster 1 setzt sich hauptsächlich aus Personen aus Frankreich, Deutschland, den Niederlanden, 
Schweden, dem Vereinigten Königreich und den USA zusammen.
Cluster 2 besteht nahezu ausschließlich aus Personen aus Brasilien.

\begin{table}[H]
    \centering
    \caption{Clustering von Tech Company}
    \label{tab:clustering_tech_company}
    \begin{tabular}{|l|l|l|}
        \hline
        \textbf{Cluster 0} & 0.91 & stark überdurchschnittlich \\ \hline
        \textbf{Cluster 1} & 0.17 & durchschnittlich \\ \hline
        \textbf{Cluster 2} & -1.07 & stark unterdurchschnittlich \\ \hline
    \end{tabular}
\end{table}
\noindent Cluster 0 arbeitet überwiegend in Tech-Unternehmen, während Cluster 2 klar unterdurchschnittlich 
im Tech-Sektor beschäftigt ist. Cluster 1 liegt hier im Durchschnittsbereich.

\begin{figure}[H]
    \centering
    \includegraphics[width=0.5\textwidth]{pics/gender_per_cluster.png}
    \caption{Geschlecht pro Cluster}
    \label{fig:gender_per_cluster}
\end{figure}
\noindent Cluster 0 weist eine hohe Konzentration an Frauen und nicht-binären Personen auf.
Cluster 2 besteht überwiegend aus Männern, während Cluster 1 eine nahezu ausgeglichene 
Geschlechterverteilung aufweist.

\begin{figure}[H]
    \centering
    \includegraphics[width=0.6\textwidth]{pics/relative_remote_work.png}
    \caption{Remote Work pro Cluster (relativ)}
    \label{fig:remote_work_per_cluster}
\end{figure}
\noindent Cluster 0 und 1 berichten überwiegend, nie remote zu arbeiten, während Cluster 2 deutlich
häufiger im Home-Office arbeitet.

\begin{figure}[H]
    \centering
    \includegraphics[width=0.6\textwidth]{pics/sharing_mental_illness.png}
    \caption{Über MH-Problemen mit Familie teilen pro Cluster (relativ)}
    \label{fig:sharing_mh_issues_per_cluster}
\end{figure}
\noindent Cluster 0 und 1 würden potenzielle mentale Probleme selten mit Familie oder Freunden 
teilen, während Cluster 2 ein ausgeglicheneres Kommunikationsverhalten zeigt.

\begin{figure}[H]
    \centering
    \includegraphics[width=0.6\textwidth]{pics/problems_at_work_treated_effectively.png}
    \caption{Schwierigkeiten in der Arbeit bei guter Behandlung}
    \label{fig:problems_at_work_treated_effectively}
\end{figure}
\begin{figure}[H]
    \centering
    \includegraphics[width=0.6\textwidth]{pics/problems_at_work_not_treated_effectively.png}
    \caption{Schwierigkeiten in der Arbeit bei schlechter Behandlung}
    \label{fig:problems_at_work_not_treated_effectively}
\end{figure}
\noindent Die Ergebnisse zeigen, dass eine ineffektive Behandlung mentaler Probleme am 
Arbeitsplatz deutlich häufiger zu arbeitsbezogenen Schwierigkeiten führt als eine effektive 
Unterstützung.

\section{Übertragung auf HR-Kontext}
Die Ergebnisse der Clusteranalyse zeigen drei klar voneinander abgegrenzte Gruppen von 
Mitarbeitenden, die sich sowohl in ihren Belastungsfaktoren als auch in ihren Ressourcen 
deutlich unterscheiden. Cluster 0 umfasst Personen, deren Belastungen vor allem in individuellen 
Faktoren verankert sind. Sie weisen einen gering ausgeprägten mentalen Gesundheitsstatus, eine 
niedrige Offenheit im Umgang mit psychischen Belastungen sowie ein reduziertes soziales 
Unterstützungsverhalten auf. Charakteristisch sind zudem Tätigkeiten im HR-, Design- oder 
administrativen Bereich sowie eine überwiegend stationäre Arbeitsweise. Diese Ausprägungen 
deuten auf einen Bedarf an gezielten Maßnahmen zur individuellen Unterstützung hin, insbesondere 
durch intensivere Mental-Health-Angebote, eine Stärkung der Führungskompetenzen zur Erkennenung
früherer Warnsignale, einen erleichterten Zugang zu Unterstützungsprogrammen sowie eine Erhöhung 
flexibler Arbeitsoptionen.\\
\\
Cluster 1 bildet eine stabile Vergleichsgruppe mit leicht überdurchschnittlichen Werten in den 
zentralen Variablen und einer ausgewogenen demografischen Verteilung. Die hier vertretenen 
Rollen stammen überwiegend aus DevOps und Engineering. Für diese Gruppe steht weniger der Bedarf 
nach neuen Interventionen im Vordergrund, sondern vielmehr die Stabilisierung der bestehenden 
Strukturen sowie ein kontinuierliches Monitoring möglicher Belastungsentwicklungen. Darüber 
hinaus bietet dieses Cluster Potenzial für den Transfer erfolgreicher Praktiken auf die beiden 
anderen Gruppen.\\
\\
Cluster 2 weist demgegenüber eine hohe Offenheit und psychische Stabilität auf. Die Mitglieder
dieser Gruppe erleben jedoch gleichzeitig ein ausgeprägtes berufliches Stigma und eine 
vergleichsweise geringe Arbeitgeberunterstützung. Diese Kombination ist bemerkenswert und weist 
auf systemische Defizite in Unternehmenskultur, Kommunikation oder vorhandenen 
Unterstützungsstrukturen hin. Die ausgeprägte psychische Stabilität könnte zudem mit der 
länderspezifischen Zusammensetzung dieses Clusters zusammenhängen, da der nahezu vollständige 
Fokus auf Brasilien darauf hindeutet, dass kulturell verankerte Verhalten und Denkweisen eine 
Rolle spielen könnten. Trotz dieser individuellen Ressourcen fehlen verlässliche organisationale 
Rahmenbedingungen, die einen offenen Umgang mit mentalen Themen ermöglichen. Folglich liegt der 
Schwerpunkt des Bedarfs auf dem Ausbau wirksamer Mental-Health-Policies sowie der Etablierung 
klarer Prozesse für Gespräche über psychische Gesundheit.\\
\\
Die Gegenüberstellung der drei Cluster verdeutlicht zudem, dass das Vorhandensein unternehmerischer 
Unterstützungsprogramme allein keine Garantie für deren Wirksamkeit darstellt. In Cluster 0 
bestehen Unterstützungsangebote, dennoch weisen die Mitarbeitenden dort die schwächste mentale 
Gesundheit und die geringste Offenheit auf. Cluster 2 hingegen verfügt über kaum institutionelle 
Unterstützung, zeigt jedoch eine vergleichsweise stabile psychische Lage. Diese Konstellation 
legt nahe, dass Programme nur dann Wirkung entfalten können, wenn sie in eine passende 
Kultur eingebettet sind und tatsächlich zugänglich und bedarfsgerecht genutzt werden.\\













