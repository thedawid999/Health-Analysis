\chapter{Datenbeschreibung und EDA}

\section{Herkunft und Struktur des Datensatzes}
- Quelle (z. B. Kaggle OSMI Mental Health in Tech 2016)\\
- Stichprobe, Anzahl der Merkmale, Datentypen\\
- Besonderheiten: Freitextfelder, kategoriale Felder, sensible Daten\\

\section{Erste deskriptive Analysen???}
- Verteilungen wichtiger Merkmale\\
- Häufigkeiten, zentrale Tendenzen\\
- Identifikation möglicher Probleme: Outlier, Inkonistenzen\\

\section{Explorative Visualisierungen}
- Histogramme, Barplots, Boxplots\\
- Korrelationen / Heatmaps\\
- Erste Hypothesen über Muster im Datensatz\\



\chapter{Datenvorverarbeitung}

\section{Umgang mit fehlenden Werten}
- Identifikation der fehlenden Werte\\
- Strategien (z. B. Dropping, Imputation, Domain-Knowledge)\\
- Begründung der gewählten Methode\\

\section{Bereinigung unstandardisierter Texteingaben???}
- Vereinheitlichung von Kategorien\\
- Lowercasing, Mapping, Domain-basierte Zusammenführung\\
- Umgang mit Freitext-Antworten\\

\section{Kodierung und Transformation der Merkmale???}
- One-Hot-Encoding, Ordinal Encoding, ggf. Target-Encoding\\
- Skalierung (Transformation)\\
- Herausforderungen bei hochkardinalen Features\\



\chapter{Feature Engineering}

\section{Feature Selection}
- Variance Threshold\\
- Korrelationen / Redundanz\\
- Relevanzbasierte Auswahl (Mutual Information)\\

\section{Feature Generation}
- Erstellen neuer Merkmale aus bestehenden Variablen\\
- Beispiele: Stress-Score, Support-Index, Arbeitsumfeld-Indikatoren\\
- Nutzen für Modellverständlichkeit und Clustering\\



\chapter{Dimensionsreduktion}
Warum Dimensionsreduktion?\\
Vorgehensweise\\

\section{Methoden der Dimensionsreduktion}
- PCA (linear)\\
- MDS, LLE (nichtlinear)\\
- Vergleich und Begründung der Auswahl\\

\section{Ergebnisse und Visualisierung}
- Erklärte Varianz (PCA)\\
- 2D/3D-Darstellungen\\
- Herausgearbeitete Muster und Trends\\



\chapter{Clustering}

\section{Auswahl geeigneter Methoden}
- K-Means\\
- Agglomeratives Clustering\\
- DBSCAN/HDBSCAN für komplexe Strukturen\\
- Begründung der Auswahl\\

\section{Bestimmung der Clusteranzahl}
- Elbow-Methode\\
- Silhouette Score\\
- Weitere Metriken\\

\section{Ergebnisse}
- Visualisierungen der Cluster (PCA/UMAP Scatterplots)\\
- Profiling: Beschreibung der typischen Merkmale jedes Clusters\\
- Identifikation gefährdeter Gruppen und Muster\\

\section{Übertragung auf HR-Kontext}
- Welcher Cluster ist besonders belastet?\\
- Welche Kombinationen von Faktoren treten gehäuft auf?\\
- Welche Gruppen könnten gezielte Unterstützung benötigen?\\













