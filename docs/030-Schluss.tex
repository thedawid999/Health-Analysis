\chapter{Schluss}

Das gewählte methodische Vorgehen hat sich insgesamt als positiv erwiesen. Besonders gut
funktionierten die Datenaufbereitung sowie das Clustering. Die klare Strukturierung 
entlang des klassischen Data-Science-Workflows 
ermöglichte eine nachvollziehbare Analyse und eine solide Grundlage für die spätere 
Interpretation. Auch die Integration neuer Merkmale erwies sich als 
hilfreich, um ein grobes Gesamtbild zu erhalten.
Herausfordernder war hingegen der Umgang mit der hohen Dimensionalität des Datensatzes und den 
zahlreichen fehlenden Werten. Trotz sorgfältiger Imputationsmethoden bleibt die Gefahr bestehen, 
dass bestimmte Muster dadurch abgeschwächt oder verstärkt wurden. Alternative Ansätze wie 
robustere Feature-Selection-Methoden,
nichtlineare Dimensionsreduktion oder andere Clusterverfahren hätten zusätzliche Perspektiven 
eröffnet und möglicherweise feinere Strukturen sichtbar gemacht.\\
\\
\noindent Die Analyse unterliegt mehreren Einschränkungen. Erstens ist die Qualität der Umfragedaten 
heterogen. Subjektive Selbsteinschätzungen sind anfällig für Verzerrungen, und der Anteil 
fehlender Werte erschwert eine zuverlässige Interpretation. Zweitens ist die Generalisierbarkeit 
der Ergebnisse begrenzt, da die Stichprobe in einigen Clustern stark von spezifischen Regionen 
oder Berufsgruppen geprägt ist. Drittens wurden 
einige potenziell relevante Faktoren nicht berücksichtigt, darunter organisational-historische 
Entwicklungen oder externe Belastungsfaktoren außerhalb des 
Arbeitsplatzes. Diese könnten zukünftige Analysen weiter differenzieren.\\
\\
\noindent Die Clusteranalyse identifizierte drei klar voneinander abgegrenzte Mitarbeitendengruppen. 
Cluster 0 weist ein personenzentriertes Belastungsprofil mit niedriger psychischer Gesundheit, 
geringer Offenheit und eingeschränkter sozialer Unterstützung auf. Cluster 1 bildet eine stabile 
Vergleichsgruppe mit leicht positiven Ausprägungen in nahezu allen Kernvariablen sowie einer 
ausgewogenen demografischen Struktur. Cluster 2 zeigt ein organisationszentriertes 
Belastungsmuster, geprägt durch hohe Offenheit und psychische Stabilität, jedoch gleichzeitig 
sehr schwache Arbeitgeberunterstützung und stark ausgeprägtes berufliches Stigma. Auffällig ist, 
dass diese psychische Stabilität möglicherweise mit der Länderzusammensetzung zusammenhängt, da 
dieses Cluster nahezu ausschließlich aus Personen aus Brasilien besteht.
Besonders problematisch ist die Feststellung, dass vorhandene Unterstützungsprogramme nicht 
automatisch wirksam sind. Cluster 0 erhält zwar prinzipiell betriebliche Angebote, befindet 
sich jedoch in der schlechtesten mentalen Lage. Cluster 2 hingegen verfügt kaum über 
Unterstützung, zeigt aber dennoch hohe psychische Stabilität.\\
\\
\noindent Für zukünftige Arbeiten bietet sich die Integration weiterer Datenquellen wie Leistungsmetriken 
oder Teamstrukturen an, um ein übersichtlicheres Bild der 
Belastungslagen zu erhalten. Ebenso wäre ein kontinuierliches Monitoring sinnvoll, um die 
Wirksamkeit spezifischer Interventionen zu prüfen und Veränderungen im Zeitverlauf sichtbar zu 
machen. Darüber hinaus eröffnet der Einsatz fortgeschrittener ML-Modelle
hohes Potenzial für ein präventiv ausgerichtetes Gesundheitsmanagement.



